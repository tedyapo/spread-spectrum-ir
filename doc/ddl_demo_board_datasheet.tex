\documentclass{article}
\usepackage{amsmath,amssymb,amsfonts} % Typical maths resource packages
\usepackage{moresize}
\usepackage{microtype}
\usepackage{fancyhdr}
\usepackage{mdwlist}
\usepackage[includeheadfoot, top=0in, bottom=0.5in, left=0.5in, right=0.5in, head=0.6in]{geometry}
\usepackage{tikz}
\usepackage{lastpage}
\usepackage{tabularx}
\usepackage{caption}
\usepackage{multirow}
\usepackage{relsize}
\usepackage{enumitem}
\usepackage{placeins}
\usepackage{float}
\usepackage{multicol}
\usepackage{siunitx}
\usepackage{hyperref}
\usepackage[printwatermark]{xwatermark}

\usepackage[compact]{titlesec}
\titlespacing{\section}{0pt}{*1}{*0}
\titlespacing{\subsection}{0pt}{*1}{*0}
\titlespacing{\subsubsection}{0pt}{*1}{*0}

\captionsetup{font={Large,bf}, labelfont={Large,bf},belowskip=0pt,aboveskip=0pt}

%\newcommand{\ddl}{DDL$^{\textsf{\textsmaller[2]{TM}}}$ }
\newcommand{\ddl}{DDL }
\newcommand{\partnumber}{SSIR01}
%\newcommand{\partdesc}{DDL$^\textsf{TM}$ Hex NOR Gate}
\newcommand{\partdesc}{Spread-Spectrum IR Proximity Sensor}
\newcommand{\releasedate}{March 2018}
\newcommand{\revisiondate}{}
%\newcommand{\revisiondate}{; Revised May 2016}

\newcommand{\qnbar}[1]{$\overline{\textsf{Q}_{#1}}$}
\newcommand{\vrf}{V$_{\textsf{RF}}$}
\newcommand{\vb}{V$_{\textsf{B}}$}
\newcommand{\tPHL}{$\textsf{t}_\textsf{PHL}$}
\newcommand{\tPLH}{$\textsf{t}_\textsf{PLH}$}
\newcommand{\VIL}{$\textsf{V}_\textsf{IL}$}
\newcommand{\VIH}{$\textsf{V}_\textsf{IH}$}
\newcommand{\VOL}{$\textsf{V}_\textsf{OL}$}
\newcommand{\VOH}{$\textsf{V}_\textsf{OH}$}
\newcommand{\tr}{$\textsf{t}_\textsf{r}$}
\newcommand{\tf}{$\textsf{t}_\textsf{f}$}
\newcommand{\tw}{$\textsf{t}_\textsf{w}$}
\newcommand{\IIL}{$\textsf{I}_\textsf{IL}$}
\newcommand{\IIH}{$\textsf{I}_\textsf{IH}$}
\newcommand{\IOS}{$\textsf{I}_\textsf{OS}$}
\newcommand{\fMAX}{$\textsf{f}_\textsf{MAX}$}
\newcommand{\comment}[1]{}

% title text for logic tables
\newcommand{\lttitle}[1]
{
  \textbf{\large{#1}}
}

% column header text for logic tables
\newcommand{\lthdr}[1]
{
  \textbf{\textsf{#1}}
}

% rising edge for logic tables
\newcommand{\risingedge}
{
\begin{tikzpicture}[scale=0.2]
\draw (0,0) -- (1,0);
\draw (1,0) -- (1.2,1);
\draw (1.2,1) -- (2.2,1);
\end{tikzpicture}
}

% falling edge for logic tables
\newcommand{\fallingedge}
{
\begin{tikzpicture}[scale=0.2]
\draw (0,1) -- (1,1);
\draw (1,1) -- (1.2,0);
\draw (1.2,0) -- (2.2,0);
\end{tikzpicture}
}

\begin{document}

%\setlength{\parskip}{0pt}
%\setlength{\parsep}{0pt}
%\setlength{\headsep}{0pt}
%\setlength{\topskip}{0pt}
%\setlength{\topmargin}{0pt}
%\setlength{\topsep}{0pt}
%\setlength{\partopsep}{0pt}

\newwatermark[allpages,align=center,angle=45,fontsize=1.25in,fontfamily=\sfdefault]{\textbf{PRELIMINARY}}

\fancypagestyle{foo}{%
\fancyhf{}
\fancyhead[L]{
\vspace*{3em}
%
% Z0 Labs Document logo
%
\noindent\begin{minipage}{0.25in}
\includegraphics[width=\textwidth]{hadio_logo.png}
%\raisebox{1em}{
%\noindent\HUGE $\boldsymbol{\mathcal{Z}\hspace{-0.225em}_\circ}$
%\LARGE
%\fontfamily{lcmss}\fontseries{b}\selectfont
%\raisebox{0.125em}{\hspace{-0.45em}LABS}
%\fontfamily{\familydefault}\fontseries{\seriesdefault}\selectfont
%}
\end{minipage}
}

\fancyhead[R]{
\LARGE \partnumber
}

\fancyfoot[L]{\small \releasedate \revisiondate}
\fancyfoot[C]{\thepage/\pageref{LastPage}}
\fancyfoot[R]{\small \copyright \, 2018 Ted Yapo. Released under CC BY 4.0 license.}

\renewcommand{\headrulewidth}{1pt}
\renewcommand{\footrulewidth}{1pt}
}

\fancypagestyle{bar}{%
\fancyhf{}

\fancyhead[L]{
\vspace*{3em}
\noindent\begin{minipage}{0.25in}
\includegraphics[width=\textwidth]{hadio_logo.png}
%
% Z0 Labs Document logo
%
%\noindent\begin{minipage}{3in}
%\raisebox{1em}{
%\noindent\HUGE $\boldsymbol{\mathcal{Z}\hspace{-0.225em}_\circ}$
%\LARGE
%\fontfamily{lcmss}\fontseries{b}\selectfont
%\raisebox{0.125em}{\hspace{-0.45em}LABS}
%\fontfamily{\familydefault}\fontseries{\seriesdefault}\selectfont
%}
\end{minipage}
}

\fancyhead[R]{
\LARGE \partnumber
}

\fancyfoot[L]{\small \releasedate \revisiondate}
\fancyfoot[C]{\thepage/\pageref{LastPage}}
\fancyfoot[R]{https://hackaday.io/ted.yapo}
%\fancyfoot[R]{
%z\hspace{0.08em}%
%\begin{tikzpicture}[scale=0.41]
%  \draw [line width=0.065em] (0.5em, 0.5em) circle [radius=0.5em];
%  \draw [line width=0.065em] (0.14645em, 0.14645em) -- (0.85355em, 0.85355em);
%\end{tikzpicture}%
%\hspace{0.08em}%
%labs.com
%}

\renewcommand{\headrulewidth}{1pt}
\renewcommand{\footrulewidth}{1pt}
}

\pagestyle{bar}
\thispagestyle{foo}

\renewcommand{\familydefault}{\sfdefault}
\fontfamily{lcmss}\fontseries{b}\selectfont
%\sisetup{detect-all}
\sisetup{math-rm=\mathsf, text-rm=\sffamily}
%%%------------------

\Huge
\noindent\textbf{\partnumber}

\vspace{0.1in}
\large
\noindent\textbf{\partdesc}

\vspace{-0.05in}
\noindent\rule{\textwidth}{1pt}
\normalsize

\vspace{0.25in}


\noindent \begin{minipage}[t][0.45\textwidth]{0.45\textwidth}
\section*{Description}
The SSIR01 is an infrared (IR) proximity sensor using direct-sequence
spread spectrum techniques to achieve low probability of transmitted or
received intereference with other IR devices.

\vspace{0.1in}
An on-board power supply generates the required RF power from four AA
batteries.

\end{minipage}
\hspace{0.05\textwidth}
\begin{minipage}[t][0.45\textwidth]{0.45\textwidth}
\section*{Features}
\setitemize[0]{itemindent=0pt,leftmargin=10pt}
\begin{itemize*}
\item Digital logic using only diodes
\item Unique circuit technology
\item Self-contained executive desk toy
\item Stores a single bit of memory
\end{itemize*}
\vfill
\end{minipage}

\vspace{0.1in}

\newcommand{\ecrow}[9]
{
  \multirow{3}{*}{#1} & \multirow{3}{*}{#2} & #4 & & #5 & & \multirow{3}{*}{#3} \\
  \cline{3-6}
  & & #6 & & #7 & & \\
  \cline{3-6}
  & & #8 & & #9 & & \\
}

\captionsetup{singlelinecheck=off}
\begin{figure}[h!]
\captionof*{figure}{Electrical Characteristics}
\centering
{\renewcommand{\arraystretch}{1.2}
\newcolumntype{C}{>{\centering\arraybackslash}p{2cm}}
\noindent\begin{tabularx}{\textwidth}{ C | X | c | c | c | c | c}
  \hline
  \textbf{Symbol} & \textbf{Parameter} & \textbf{Conditions} & \textbf{Min.} &
  \textbf{Typ.} & \textbf{Max.} &  \textbf{Units} \\ \hline
  V$_\textsf{batt}$ & Battery supply voltage & & 2.0 & 4.8 & 6.5 &\si{\volt}\\
  \hline
  I$_\textsf{batt}$ & Battery current drain & & & 50 & &\si{\milli \ampere}\\
  \hline
  f$_\textsf{supp}$ & RF power supply frequency & & & 3.6 & &\si{\mega \hertz}\\
  \hline
  --- & Data storage capacity & & & 1 & &bit\\
  \hline
  --- & Battery lifetime & & & 48 & &\si{hours}\\
  \hline
\end{tabularx}}

\raggedright
\begin{tabular}{r l}
Notes: & 1. Typical measurements with red LEDs at Vbatt = 5.0V \\
       & 2. Battery lifetime based on 2000 mAH NiMH cells
\end{tabular}

\end{figure}
\captionsetup{singlelinecheck=on}


\FloatBarrier
\clearpage

\begin{figure}[htbp]
\centering
%\includegraphics{DDL_demo_board_diagram.pdf} \\
\vspace{8pt}
\caption{DDL Demo Board (1:1 scale on US letter size paper)}
\label{fig:diagram}
\end{figure}

\begin{figure}[htbp]
\centering
%\includegraphics{ddl_demo_board_latch_schematic.pdf} \\
\vspace{8pt}
\caption{DDL Demo Board Latch Schematic}
\label{fig:latch}
\end{figure}

\begin{figure}[htbp]
\centering
%\includegraphics{ddl_demo_board_power_supply_schematic.pdf} \\
\vspace{8pt}
\caption{DDL Demo Board Power Supply Schematic}
\label{fig:supply}
\end{figure}


\setlength{\columnsep}{0.25in}
\begin{multicols}{2}


\section*{Operation}
\noindent A diagram of the DDL Demo Board is shown in Figure
\ref{fig:diagram}. The user interface consists of a power switch, a
pair of Set/Reset pushbutton switches, and two indicator LEDs.

\subsection*{Batteries}
\noindent The DDL Demo Board requires (4) IEC-LR6 (AA size) batteries.  The use
of NiMH batteries is recommended, although NiCd, alkaline or lithium
primary batteries may also be used.

Rechargeable batteries should be replaced before they are completely
discharged to preserve battery health. The DDL Demo Board is capable
of completely draining batteries, which may negatively impact the
number of charge/discharge cycles possible with rechargeable types.

\subsection*{Power Switch}
\noindent Use the power switch to turn the DDL Demo Board on or off.  Note that
the state of the board (storing a 0 or 1) is lost on power off.  Upon
power-up, the board may assume either one of the states, although
most boards will show a bias toward one of the states due to slight
differences in component values in the latch circuit.

\subsection*{Data Storage}
\noindent Upon power up, the board will assume one of the states (0 or
1).  To set a specific state, press one of the white buttons (Set or
Reset).  The circuit will continue to hold the desired state until the
power is turned off.

\subsection*{RF Emissions}
The DDL Demo Board may emit unintentional RF radiation between 3 and 4
MHz, and at harmonics thereof.  In case this causes interference with
any other electronic device, switch off the power to the DDL Demo
Board.

\section*{\ddl Background}
\noindent \ddl gates implement standard logic functions using silicon
diodes as active elements.  In order to enable arbitrary logic
functions, \ddl gates must provide absolute gains greater than unity
and a logical inversion function, both of which are absent from
previous diode-based logic. To achieve the required gain and inversion
functions, PIN-type diodes are used to switch power from a
radio-frequency (RF) supply, \vrf. The switched RF power is then
rectified by a voltage doubler consisting of 1N4148 diodes and LEDs to
produce a DC output signal. On the DDL Demo Board, common rectifier
diodes (1N4007 typ.)  are used as PIN diodes, achieving good
performance at very low cost.

\subsection*{Latch Design}
\noindent A schematic diagram of the latch circuit implemented on the
DDL Demo Board is shown in Figure \ref{fig:latch}.  Two diode-based
inverters are cross-coupled to form an RS latch.  In operation, an RF
power supply of nominally 5V amplitude at 3.6MHz is used as \vrf.
Assuming the latch is in the set state, D1 has negligible DC current
flowing through it, so it assumes a high impedance at the RF power
supply frequency. This allows RF power from the supply to flow through
R2 and C1 to the voltage doubler formed by C2, D2, D3, and C3.  The
resulting DC output current flows through L2 and D4, causing D4 to
assume a low impedance, shunting RF current away from the lower
voltage doubler consisting of C5, D5, D6, and C6.  Since little
current is allowed to flow in the lower doubler, a low DC bias is
maintained on D1, completing the feedback loop.  The result is a
stable state with the Q LED illuminated.

With the latch in the reset state, the roles of the upper and lower
inverters are reversed, with D1 shunting RF current to ground, and
D4 allowing current to flow through the lower voltage doubler.

When pressed, push-button switches SW1 and SW2 cause the latch to
assume either the RESET or SET state, respectively, buy forcing either
D1 or D4 into a low RF impedance state.  When the switch input is
removed, the latch retains its current state.

\subsection*{RF Power Supply}
\noindent The DDL latch on the DDL Demo Board requires an RF power
supply for operation.  The simple supply shown in Figure
\ref{fig:supply} is implemented on the board.  In operation, R4, C7,
and U1a, a Schmitt-trigger inverter, form an RC relaxation oscillator
with a nominal frequency around 3.6 MHz.  The resulting signal is
buffered through U1b to drive a push-pull output stage consisting of
U1c-U1f.  Output resistors R5-R8 establish an output impedance of
50$\Omega$ for the output stage, which is AC-coupled to output
transformer T1.  T1 acts as a balun transformer converting the
balanced output of the push-pull stage to unbalanced RF power for the
latch circuit.

\end{multicols}
\end{document}

